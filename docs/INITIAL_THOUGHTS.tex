Following are some initial thoughts on ring like trust relationships
over secure internet data paths established without a third party
certificate authority involved.

Social aspects of network traffic.

Let's say you want communicate with your peer over the internet. And
let's say, you want to use a secure connection, a secure channel to do
so.  The usual practices used today to establish such a secure channel
are either involving a third party (a CA, a certificate authority), or
using OOB (out of band) practices to (pre)share a secret token to
ensure there sits no MITM (man in the middle) during secure connection
protocol negotiation. Think of DH (Diffie Hellman key exchange) for
example.

The benefits and the backside of CA infrastructure based solutions are
already discussed in length. In the last time OOB practices are used
to add additional security to CA based solutions. Checking a digital
signature or hash value of a downloaded install package is an OOB
operation. The ssh shell presenting unknows fingerprints to the user
asking for proof, and then to be added to the known hosts file is
another one.

We want to give the open brain a chance to think about another use
case. A new way of thinking about network traffic in general. A new
way in thinking about what the Internet provides to us, how we usually
see it and how we make use of it. Or better the way we usually do not.


Speaking in simple technical terms, the internet is nothing more than
a huge bunch of loosely interconnected hosts, or nodes. From the view
point of a data packet travelling around, there are only differences
in the capacity of the bandwith involved on the interconnections, or
the computing power and software running on the nodes. May that be
backbone switches, routers, servers, fibre cable, coper wire, what
ever.

Some important base principles of the internet are:

- from the address point of view, every host can reach every other
host, often with redundant data paths involved.
- net neutrality. Every data packet has the same right to be transported.
- every host can act as a server, or a client, or both.

Let's take a usual use case of using the internet as an endpoint
user. You have your software app on your smartphone, and you send a
message to a friend over one of the well known message transporting
provider company networks. Those provider company networks provide
your smartphone with a direct connection to the internet, usually over
RF.

Let's analyze from a data packet and address point of view the message
sending process. Your message is intented to be received at the
internet connected device of you peer.

In the case the designated peer to receive the message is connected to
the internet, too, there is always a more or less direct path to its
internet device already. There is no need to store the message on a
third party server in this case. In particular you would have gotten
an destination host unreachable protocol error reply in that case. If
not, the message reaches the recipient instantly and on a direct
connection involved - would we at all make use of it.

With your peers, your firends, your social contacts on the internet,
there comes the social aspect. What if i say to my known friends, my
known contact list, 'hey, i'm online again, send me your queued
messages!', when i activate my device ? Do i need a server involved
for that ? No. I only need the actual network addresses of my peers.

Network addresses for end user devices, like dial up connection
addresses, too, are usually dynamically allocated. With the IoT
movement and IP6 address space dilatation, this may no longer be
needed. But we have to take it as it is for now.

So when i activate my device, i want to tell my firends my new,
dynamically allocated address, so they can send me their messages,
content, comments directly. But wait, what if their addresses
changed, too ?

It looks, at a first glance, there is no way around a centralized
structure like the today widespread usual cleint/server network
topology. We are used to have (and accept), that there is allways an
authority ABOVE, a MASTER to help us, to connect us to our friends,
tell us their changed addresses, provide us with legal notices about
OUR content (how silly is that ? Just today i've got a message from
youblood telling me THEY changed the terms of use, i.e. how the
wording of MY license changed i gave to THEM, regarding MY content
!!!), filter it and so on.

The one who can imagine a solution to this problem, without reading
further, is usually way ahead of what we have implemented today.

Let's start again with our two peers. Both have their devices turned
on and connected to the internet. There is a direct data path. Both
download an open source based app from an app store. This app is able
to connect directly to any other arbitrary internet address. In this
case, the two users make a phone call (OOB), share a secret token with
each other over the out of band telephone transport layer, which
allows it to establish a fine cryptographically secure two way
connection to each other. This connection can stay as long as both
users have turned on their devices and there is no interruption due to
new internet address negotiation or whatever.

The social aspect of this secure internet connection is, that both
peers trust each other. Both share the trust that the other peer is
who the other peer claims to be, (usually CA signed certificate
authenticity) end does no bad with the secure channel established
(Usual trust in the other end. When i connect over https to my bank, i
have trust in what they do with my data send over the secure
channel). Why should i establish a secure connection when the other
end is known to be untrustful already ?

So did both peers not share any trust at all, they woulnd'nt have
established a connection at all.

As the connection is cryptographically secure, they can share private
and sensible data without fear of beeing seen by unwanted third
parties. Thats usually a good thing! Think of the last time you had
really good words with a friend, having a private conversation. There
was trust involved, too. Now you have a secure two point connection
over the internet to do the same.

Lets take a third person, a friend of one of the both already having a
secure, trustworthy channel. This third person takes one of the
starting two, and makes the same two way secure connection to one of
the two already involved. We got a V shape secure connection
structure, with three nodes. Two at the end, and one in the
middle. When you visualize a man in the middle attack, this is exactly
what we have now, but as we have cryprographically secure connections
without any third party involved (remember, OOB pre shared secret
token connection negotiation), we use the man in the middle scheme to
our benefit. We turn something bad into something good, which is
usually also a good thing !

Let's say further, the both single connected nodes at the end of the V
structure want to establish a direct, secure connection, too. Do they
have to pre share a secure token OOB ? Think twice. The answer is
yes. But there is a secure data path already available over the
friendly man in the middle node! No phone call necessary !!

We end up with a ring or circle shaped structure with two way
connections all around. Everything what happens inside this simple
ring of trust is based on trust. One person trust another one and
accepts a secure pre shared token. A secure connection establishes. A
third person gains trust two on of the two involved, and a two-way
connection ring like structure forms - involving only direct data
paths over the internet.

So long, so simple. This simple approach can easily be extended. Think
of a ring of 5, 10, 20 nodes, all directly connected in a ring like
structure. This is how the nature arranges atoms in our molecules. How
big can a ring expand until it breaks apart ? In how many rings can a
user have nodes in to participate ? What when data travelles in one
ring, and on every node gets shared to the other rings the user holding
the node is connected to ?

Here are some key aspects to think about further:

A ring like structure (think of Token Ring network topology) has imminent positive properties:

The integrity can be assured.
A point of failure can easily be detected.
When two points fail, two new rings can form.

